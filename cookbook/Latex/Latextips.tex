

%///////////////////////
%Center big pictures
%//////////////////////

			\begin{figure}[!ht] 
			\centering
			\noindent\makebox[\textwidth]{
			\includegraphics[scale=.7]{image.png}
			\caption{}
			}
			\label{}
			\end{figure}
			

%///////////////////////
%Change col size
%///////////////////////


%//header
\usepackage{array}
\newcolumntype{L}[1]{>{\raggedright\let\newline\\\arraybackslash\hspace{0pt}}m{#1}}
\newcolumntype{C}[1]{>{\centering\let\newline\\\arraybackslash\hspace{0pt}}m{#1}}
\newcolumntype{R}[1]{>{\raggedleft\let\newline\\\arraybackslash\hspace{0pt}}m{#1}}

%In the array
\begin{table}[!ht]
\begin{center}
\begin{tabular}{|c|c|c|C{5cm}|}  //
  \hline
  Name &  Ldb cut* & Efid Eheat cut** & Additional cut***  \\
  \hline
  ID2  & (1.6,2,4) & (2.5,2) & Estimator-Eheat<3,\newline Ecol1-Ecol2<3 \\
  \hline
\end{tabular}
\end{center}
\caption{}
}

%//////////////////////
%Skip a line
%////////////////////
~\\

%////////////////////
%Tilde
%////////////////////
$\widetilde{\lambda}$

%////////////////////
%Degrees
%////////////////////
$^\circ$

%///////////////////
%Calligraphy
%///////////////////
$\cal{X}$ 

%///////////////////
%Resize a table
%///////////////////
\usepackage{graphics}

\begin{table}
\centering
\resizebox{\columnwidth}{!}{%
\begin{tabular}{r|lll}
\multicolumn{1}{r}{}
& \multicolumn{1}{l}{Heading 1}
& \multicolumn{1}{l}{Heading 2}
& \multicolumn{1}{l}{Heading 3} \\ \cline{2-4}
Row 1 & Cell 1,1 & Cell 1,2 & Cell 1,3 \\
Row 2 & Cell 2,1 & Cell 2,2 & Cell 2,3
\end{tabular}%
}
\end{table}


%///////////////////
%Change the title of the table of contents (default is Contents, it is changed to Summary)
%///////////////////
%
\renewcommand*\contentsname{Summary}
\begin{document}
\maketitle
\tableofcontents


%///////////////////
%Use French
%///////////////////

\usepackage[utf8]{inputenc}
\usepackage[T1]{fontenc}
\usepackage[frenchb]{babel}


%///////////////////
%No blank page between chapters etc (remove this option for printing?) 
%///////////////////
\documentclass[11pt,a4paper,openany]{book}

%///////////////////
%Remove page numbering. Will need to be reset eg \pagenumbering{arabic} if ANY other page number is to be written
%///////////////////
\pagenumbering{gobble}

%///////////////////
%Graphic path
%///////////////////
\graphicspath{{Pictures/}} % Specifies the directory where pictures are stored


%///////////////////
%Include eps files
%///////////////////
\usepackage{epstopdf}

and modify in Options Configure PdfLatex --> Add -shell-escape


%///////////////////
%Table of figures
%///////////////////
		\begin{figure}[!ht]
		  \centering
		  \subfloat{\label{fig:BAO1}\includegraphics[width=60mm]{Figures/Chap1/BAO1.eps}}
		  \subfloat{\label{fig:BAO2}\includegraphics[width=60mm]{Figures/Chap1/BAO2.eps}}
		  \\
		  \subfloat{\label{fig:BAO3}\includegraphics[width=60mm]{Figures/Chap1/BAO3.eps}}
		  \subfloat{\label{fig:BAO4}\includegraphics[width=60mm]{Figures/Chap1/BAO4.eps}}
		  \\
		  \subfloat{\label{fig:BAO5}\includegraphics[width=60mm]{Figures/Chap1/BAO5.eps}}
		  \subfloat{\label{fig:BAO6}\includegraphics[width=60mm]{Figures/Chap1/BAO6.eps}}
		  \\
		\rule{35em}{0.5pt}
		\label{fig:BAO}
		\captionof{figure}{Roger. Figure from~\citep{SN_constraints_2014}}
		\end{figure}


%///////////////////
%Insert some code
%///////////////////
\usepackage{listings}
\usepackage{color}

\definecolor{codegreen}{rgb}{0,0.6,0}
\definecolor{codegray}{rgb}{0.5,0.5,0.5}
\definecolor{codepurple}{rgb}{0.58,0,0.82}
\definecolor{backcolour}{rgb}{0.95,0.95,0.92}
  
\lstdefinestyle{mystyle}{
    backgroundcolor=\color{backcolour},   
    commentstyle=\color{codegreen},
    keywordstyle=\color{magenta},
    numberstyle=\tiny\color{codegray},
    stringstyle=\color{codepurple},
    basicstyle=\footnotesize,
    breakatwhitespace=false,         
    breaklines=true,                 
    captionpos=b,                    
    keepspaces=true,                 
    numbers=left,                    
    numbersep=5pt,                  
    showspaces=false,                
    showstringspaces=false,
    showtabs=false,                  
    tabsize=2
}

\lstset{style=mystyle}

\begin{lstlisting}[language=C]

for (int k=0; k<num_events; k++) 
{      
  //Gaussian noise on veto
  EIA=u.Gaus(0,s_i);
  EIC=u.Gaus(0,s_i);
  //recoil_vec and Q_vec contain the Quenching 
  //and recoil energy of simulated events
  //Heat and collectrodes are reconstructed with Luke formula + noise
  EC1=(1+Q_vec[k]*V_fid/3.)/(1+V_fid/3.)*recoil_vec[k]+u.Gaus(0,s_h);
  EC2=(1+Q_vec[k]*V_fid/3.)/(1+V_fid/3.)*recoil_vec[k]+u.Gaus(0,s_h);       
  EIB=u.Gaus(0,s_i)+recoil_vec[k]*Q_vec[k];
  EID=u.Gaus(0,s_i)+recoil_vec[k]*Q_vec[k];
  double EC = 0.5*(EC1+EC2);
  double EI = 0.5*(EIA+EIB+EIC+EID);

  //This is to implement the trigger threshold effect
  double accept = 0.5*(1+TMath::Erf((EC-0.5)/(0.4*sqrt(2)/2.3548)));
  //This it to implement the cuts outlined in Sec. 1
  if (1 < EC && EC < 15 && 1<EI  && EI<15 && EIA < 5*s_i && EIC < 5*s_i u.Binomial(1,accept)==1) 
  //The training WIMP tree is filled with the 6-tuple  
  //(EIA, EIB, EIC, EID, EC1, EC2)
  t_new->Fill();
  
}

\end{lstlisting}


%///////////////////
%Citation havoc in a figure
%///////////////////
instead of 
\caption[\citep{chap3:CAST_gay}},
use
\caption[]{\protect\citep{chap3:CAST_gay}}

This is because it first starts to compile figures, then the main text (smthg like that)
So without the trick, the citation order is messed up
\begin{figure}[!ht]
\centering
\begin{subfigure}{.3\textwidth}
  \centering
  \begin{tikzpicture}[scale=0.8]
  \pie[rotate=120,color={ RoyalBlue, Apricot, Dandelion, BurntOrange}]{63/ Dark Matter,10/Neutrinos,12/Atoms, 15/Photons}
  \end{tikzpicture}}
\end{subfigure}\hfill%
\begin{subfigure}{.3\textwidth}
  \centering
  \begin{tikzpicture}[scale=0.8]
  \pie[rotate=120,color={Purple, Dandelion, RoyalBlue}]{68.3/ Dark Energy, 4.9/Ordinary Matter, 26.8/Dark Matter}
  \end{tikzpicture}
\end{subfigure}
\hspace*{\fill}
\caption{Left: Energy budget of the universe 380 000 years after Big Bang (recombination epoch). Right: Energy budget today.}
\label{fig:test}
\end{figure}


%///////////////////
%Pie chart with tikz
%///////////////////

\usepackage{color}
\usepackage[usenames,dvipsnames]{xcolor}

\usepackage{tikz}
\usepackage{pgf-pie}
\usepackage{listings}

\def\angle{0}
\def\radius{3}
\def\cyclelist{{"orange","blue","red","green"}}


\begin{figure}[!ht]
    \begin{subfigure}[c]{0.5\textwidth}
        %\centering
\newcount\cyclecount \cyclecount=-1
\newcount\ind \ind=-1
  \begin{tikzpicture}[scale=0.6,nodes = {font=\sffamily}]
    \foreach \percent/\name in {
        68.3/Dark Matter,
        10/Neutrinos,
        15/Photons,
        12/Atoms
      } {
        \ifx\percent\empty\else               % If \percent is empty, do nothing
          \global\advance\cyclecount by 1     % Advance cyclecount
          \global\advance\ind by 1            % Advance list index
          \ifnum3<\cyclecount                 % If cyclecount is larger than list
            \global\cyclecount=0              %   reset cyclecount and
            \global\ind=0                     %   reset list index
          \fi
          \pgfmathparse{\cyclelist[\the\ind]} % Get color from cycle list
          \edef\color{\pgfmathresult}         %   and store as \color
          % Draw angle and set labels
          \draw[fill={\color!50},draw={\color}] (0,0) -- (\angle:\radius)
            arc (\angle:\angle+\percent*3.6:\radius) -- cycle;
          \node at (\angle+0.5*\percent*3.6:0.7*\radius) {\percent\,\%};
          \node[pin=\angle+0.5*\percent*3.6:\name]
            at (\angle+0.5*\percent*3.6:\radius) {};
          \pgfmathparse{\angle+\percent*3.6}  % Advance angle
          \xdef\angle{\pgfmathresult}         %   and store in \angle
        \fi
      };
  \end{tikzpicture}
    \end{subfigure}\hspace{-3em}%
    \begin{subfigure}[c]{0.5\textwidth}
        %\centering
\newcount\cyclecount \cyclecount=-1
\newcount\ind \ind=-1
  \begin{tikzpicture}[scale=0.6,nodes = {font=\sffamily}]
    \foreach \percent/\name in {
        68.3/Dark Energy,
        26.8/Dark Matter,
        4.9/Ordinary Matter
      } {
        \ifx\percent\empty\else               % If \percent is empty, do nothing
          \global\advance\cyclecount by 1     % Advance cyclecount
          \global\advance\ind by 1            % Advance list index
          \ifnum3<\cyclecount                 % If cyclecount is larger than list
            \global\cyclecount=0              %   reset cyclecount and
            \global\ind=0                     %   reset list index
          \fi
          \pgfmathparse{\cyclelist[\the\ind]} % Get color from cycle list
          \edef\color{\pgfmathresult}         %   and store as \color
          % Draw angle and set labels
          \draw[fill={\color!50},draw={\color}] (0,0) -- (\angle:\radius)
            arc (\angle:\angle+\percent*3.6:\radius) -- cycle;
          \node at (\angle+0.5*\percent*3.6:0.7*\radius) {\percent\,\%};
          \node[pin=\angle+0.5*\percent*3.6:\name]
            at (\angle+0.5*\percent*3.6:\radius) {};
          \pgfmathparse{\angle+\percent*3.6}  % Advance angle
          \xdef\angle{\pgfmathresult}         %   and store in \angle
        \fi
      };
  \end{tikzpicture}
    \end{subfigure}
\rule{35em}{0.5pt}  
\caption{Left: Energy budget of the universe 380 000 years after Big Bang (recombination epoch). Right: Energy budget today.}
\label{fig:pie_chart}
\end{figure}


%///////////////////
%Pie chart with tikz + pgf-pie
%///////////////////

\usepackage{color}
\usepackage[usenames,dvipsnames]{xcolor}

\usepackage{tikz}
\usepackage{pgf-pie}
\usepackage{listings}

\begin{figure}[!ht]
\centering
\begin{subfigure}{.3\textwidth}
  \centering
  \begin{tikzpicture}[scale=0.8]
  \pie[rotate=120,color={ RoyalBlue, Apricot, Dandelion, BurntOrange}]{63/ Dark Matter,10/Neutrinos,12/Atoms, 15/Photons}
  \end{tikzpicture}}
\end{subfigure}\hfill%
\begin{subfigure}{.3\textwidth}
  \centering
  \begin{tikzpicture}[scale=0.8]
  \pie[rotate=120,color={Purple, Dandelion, RoyalBlue}]{68.3/ Dark Energy, 4.9/Ordinary Matter, 26.8/Dark Matter}
  \end{tikzpicture}
\end{subfigure}
\hspace*{\fill}
\caption{Left: Energy budget of the universe 380 000 years after Big Bang (recombination epoch). Right: Energy budget today.}
\label{fig:test}
\end{figure}

%///////////////////
%Line break in table
///////////////////
\begin{table}[!ht]
\begin{center} 
\renewcommand{\arraystretch}{1.5}
\begin{tabular}{|c|c|c|c|c|}
  \hline
  Features & EC1, EC2, EIA, EIB, EIC, EID\\
  & HR, EFID-QER \\
  \hline
  Tree depth & 3\\
  \hline
  Number of boosting rounds & 150\\
  \hline
  Learning rate & 0.1\\
  \hline
\end{tabular}
\end{center}
\rule{35em}{0.5pt}
\caption{BDT hyperparameters and feature selection.}
\label{table:bdt_param} 
\end{table}



%///////////////////
%Flow chart
%///////////////////
\begin{textblock*}{12cm}(0.5cm,-1cm)
\scalebox{0.6}{
      \begin{tikzpicture}[node distance=2.3cm]
        \onslide<2->\node[block]                  (init){Choix $g_{A\gamma}^{\rm simu}$ et $\alpha^{\rm détecteur}$};
        \onslide<3->\node[block, below of=init]   (nbrh){Simulation fond + signal};
        \onslide<4->\node[block, below of=nbrh](ovgt){Construction de $D(\tilde{\lambda}|g_{A\gamma}^{\rm simu})$};
        \onslide<5->\node[block, below of=ovgt]   (accp){Calcul de $I(g_{A\gamma}^{\rm simu})$};
        % \node[block, below of=accp]   (incr){Calcul de $I(g_{A\gamma}^{\rm simu})$};       
        \onslide<6->\node[decision, below=0.5cm of accp] (stcd){Stat suffisante ?};
        \onslide<8->\node[block, right= 2cm of stcd]   (stop){Calcul de $I(g_{A\gamma}^{\rm simu}, 95\%)$};
        \onslide<9->\node[decision, above = 7cm of stop]   (calc){\footnotesize$I(g_{A\gamma}^{\rm simu}, 95\%)$ $= I_{\rm vrai}$?};
        \onslide<11->\node[block, right=2cm of calc]   (fin){$g_{A\gamma}^{\rm lim} = g_{A\gamma}^{\rm simu}$};
        % invisible node helpful later
        \node[left=1cm of accp,scale=0.05](inv){};

        % second invisble
        \node[above=4.32cm of calc,scale=0.05](inv2){};

        \onslide<3->\path[line] (init) --          (nbrh);
        \onslide<4->\path[line] (nbrh) --          (ovgt);
        \onslide<5->\path[line] (ovgt) -- node[left]{oui}(accp);
        \onslide<6->\path[line] (accp) --          (stcd);
       \onslide<7->\path[-,draw] (stcd) -| node{} (inv.north);
       \onslide<7->\path[line]{} (inv.north) |- node[above]{non} (nbrh);
        \onslide<8->\path[line] (stcd) -- node[below]{oui}(stop);
        \onslide<9->\path[line] (stop) -- (calc);
        \onslide<10->\path[line] (calc) -- node[below]{non}(init);
        \onslide<11->\path[line] (calc) -- node[below]{oui}(fin);
       %
       % \path[-,draw] (calc) -- (inv2);
       % \path[line] (inv2) --  node[above]{non}(init);


      \end{tikzpicture}%
}      
\end{textblock*}


%///////////////////
%Nice table
%///////////////////
\begin{center}
\begin{tabular}{ | c | c | c |}
\hline
\thead{Production channel} & \thead{Détection channel} & \thead{Couplages} \\
\hline
\makecell{\textcolor{uibgreen}{\textbf{Production Primakoff}} \\\textcolor{uibgreen}{\textbf{dans le soleil}}} & \textcolor{uibgreen}{\textbf{Détection Primakoff}}  & \textcolor{uibgreen}{\textbf{$\bf g_{\bf A\bf\gamma}^4$}}  \\
\hline
\makecell{Production CBRD \\dans le soleil} & Détection axio-électrique & $g_{Ae}^4$\\
\hline
\makecell{Production $^{57}$Fe \\dans le soleil} & Détection axio-électrique & $g_{AN}^2\times g_{Ae}^2$\\
\hline
\makecell{Axion matière noire} & Détection axio-électrique & $g_{Ae}^2$\\
\hline
\end{tabular}
\end{center}